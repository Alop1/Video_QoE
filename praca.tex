\documentclass[11pt]{aghdpl}
% \documentclass[en,11pt]{aghdpl}  % praca w języku angielskim

% Lista wszystkich języków stanowiących języki pozycji bibliograficznych użytych w pracy.
% (Zgodnie z zasadami tworzenia bibliografii każda pozycja powinna zostać utworzona zgodnie z zasadami języka, w którym dana publikacja została napisana.)
\usepackage[english,polish]{babel}

% Użyj polskiego łamania wyrazów (zamiast domyślnego angielskiego).
\usepackage{polski}

\usepackage[utf8]{inputenc}

% dodatkowe pakiety

\usepackage{mathtools}
\usepackage{amsfonts}
\usepackage{amsmath}
\usepackage{amsthm}

% --- < bibliografia > ---

\usepackage[
style=numeric,
sorting=none,
%
% Zastosuj styl wpisu bibliograficznego właściwy językowi publikacji.
language=autobib,
autolang=other,
% Zapisuj datę dostępu do strony WWW w formacie RRRR-MM-DD.
urldate=iso8601,
% Nie dodawaj numerów stron, na których występuje cytowanie.
backref=false,
% Podawaj ISBN.
isbn=true,
% Nie podawaj URL-i, o ile nie jest to konieczne.
url=false,
%
% Ustawienia związane z polskimi normami dla bibliografii.
maxbibnames=3,
% Jeżeli używamy BibTeXa:
backend=bibtex
]{biblatex}

\usepackage{csquotes}
% Ponieważ `csquotes` nie posiada polskiego stylu, można skorzystać z mocno zbliżonego stylu chorwackiego.
\DeclareQuoteAlias{croatian}{polish}

\addbibresource{bibliografia.bib}

% Nie wyświetlaj wybranych pól.
%\AtEveryBibitem{\clearfield{note}}


% ------------------------
% --- < listingi > ---

% Użyj czcionki kroju Courier.
\usepackage{courier}

\usepackage{listings}
\lstloadlanguages{TeX}

\lstset{
	literate={ą}{{\k{a}}}1
           {ć}{{\'c}}1
           {ę}{{\k{e}}}1
           {ó}{{\'o}}1
           {ń}{{\'n}}1
           {ł}{{\l{}}}1
           {ś}{{\'s}}1
           {ź}{{\'z}}1
           {ż}{{\.z}}1
           {Ą}{{\k{A}}}1
           {Ć}{{\'C}}1
           {Ę}{{\k{E}}}1
           {Ó}{{\'O}}1
           {Ń}{{\'N}}1
           {Ł}{{\L{}}}1
           {Ś}{{\'S}}1
           {Ź}{{\'Z}}1
           {Ż}{{\.Z}}1,
	basicstyle=\footnotesize\ttfamily,
}

% ------------------------

\AtBeginDocument{
	\renewcommand{\tablename}{Tabela}
	\renewcommand{\figurename}{Rys.}
}

% ------------------------
% --- < tabele > ---

\usepackage{array}
\usepackage{tabularx}
\usepackage{multirow}
\usepackage{booktabs}
\usepackage{makecell}
\usepackage[flushleft]{threeparttable}

% defines the X column to use m (\parbox[c]) instead of p (`parbox[t]`)
\newcolumntype{C}[1]{>{\hsize=#1\hsize\centering\arraybackslash}X}


%---------------------------------------------------------------------------

\author{Pola Łącz}
\shortauthor{Pola Łącz}

%\titlePL{Przygotowanie bardzo długiej i pasjonującej pracy dyplomowej w~systemie~\LaTeX}
%\titleEN{Preparation of a very long and fascinating bachelor or master thesis in \LaTeX}

\titlePL{Opracowanie, analiza i ocena algorytmu wyznaczania wartości MOS}
\titleEN{Develope an Algorithm Predicting MOS as a Function of FR Metrics}


\shorttitlePL{Przygotowanie pracy dyplomowej w~systemie \LaTeX} % skrócona wersja tytułu jeśli jest bardzo długi
\shorttitleEN{Preparation of a long and fascinating thesis in \LaTeX}

\thesistype{Praca dyplomowa magisterska}
%\thesistype{Master of Science Thesis}

\supervisor{dr hab. Lucjan Janowski }
%\supervisor{Marcin Szpyrka PhD, DSc}

\degreeprogramme{Teleinformatyka}
%\degreeprogramme{Computer Science}

\date{2019}

\department{Katedra Telekomunikacji}
%\department{Department of Applied Computer Science}

\faculty{Wydział Informatyki, Elektroniki i Telekomunikacjij}
%\faculty{Faculty of Electrical Engineering, Automatics, Computer Science and Biomedical Engineering}

\acknowledgements{Serdecznie dziękuję \dots tu ciąg dalszych podziękowań np. dla promotora, żony, sąsiada itp.}


\setlength{\cftsecnumwidth}{10mm}

%---------------------------------------------------------------------------
\setcounter{secnumdepth}{4}
\brokenpenalty=10000\relax

\begin{document}

\titlepages

% Ponowne zdefiniowanie stylu `plain`, aby usunąć numer strony z pierwszej strony spisu treści i poszczególnych rozdziałów.
\fancypagestyle{plain}
{
	% Usuń nagłówek i stopkę
	\fancyhf{}
	% Usuń linie.
	\renewcommand{\headrulewidth}{0pt}
	\renewcommand{\footrulewidth}{0pt}
}

\setcounter{tocdepth}{2}
\tableofcontents
\clearpage

%\chapter{Przykłady elementów pracy dyplomowej}

\section{Liczba}

Pakiet \texttt{siunitx} zadba o to, by liczba została poprawnie sformatowana: \\
\begin{center}
	\num{1234567890.0987654321}
\end{center}


\section{Rysunek}

Pakiet \texttt{subcaption} pozwala na umieszczanie w podpisie rysunku odnośników do ,,podilustracji'': \\

\begin{figure}[h]
	\centering
	\begin{subfigure}{0.35\textwidth}
		\centering
		\framebox[2.0\width]{A}
		\subcaption{\label{subfigure_a}}
	\end{subfigure}
	\begin{subfigure}{0.35\textwidth}
		\centering
		\framebox[2.0\width]{B}
		\subcaption{\label{subfigure_b}}
	\end{subfigure}
	
	\caption{\label{fig:subcaption_example}Przykład użycia \texttt{\textbackslash subcaption}: \protect\subref{subfigure_a} litera A, \protect\subref{subfigure_b} litera B.}
\end{figure}

\section{Tabela}

Pakiet \texttt{threeparttable} umożliwia dodanie do tabeli adnotacji: \\

\begin{table}[h]
	\centering
	
	\begin{threeparttable}
		\caption{Przykład tabeli}
		\label{tab:table_example}
		
		\begin{tabularx}{0.6\textwidth}{C{1}}
			\toprule
			\thead{Nagłówek\tnote{a}} \\
			\midrule
			Tekst 1 \\
			Tekst 2 \\
			\bottomrule
		\end{tabularx}
		
		\begin{tablenotes}
			\footnotesize
			\item[a] Jakiś komentarz\textellipsis
		\end{tablenotes}
		
	\end{threeparttable}
\end{table}

\section{Wzory matematyczne}

Czasem zachodzi potrzeba wytłumaczenia znaczenia symboli użytych w równaniu. Można to zrobić z użyciem zdefiniowanego na potrzeby niniejszej klasy środowiska \texttt{eqwhere}.

\begin{equation}
E = mc^2
\end{equation}
gdzie
\begin{eqwhere}[2cm]
	\item[$m$] masa
	\item[$c$] prędkość światła w próżni
\end{eqwhere}

Odległość półpauzy od lewego marginesu należy dobrać pod kątem najdłuższego symbolu (bądź listy symboli) poprzez odpowiednie ustawienie parametru tego środowiska (domyślnie: 2 cm).

%\chapter{Wprowadzenie}
\label{cha:wprowadzenie}

\LaTeX~jest systemem składu umożliwiającym tworzenie dowolnego typu dokumentów (w~szczególności naukowych i technicznych) o wysokiej jakości typograficznej (\cite{Dil00}, \cite{Lam92}). Wysoka jakość składu jest niezależna od rozmiaru dokumentu -- zaczynając od krótkich listów do bardzo grubych książek. \LaTeX~automatyzuje wiele prac związanych ze składaniem dokumentów np.: referencje, cytowania, generowanie spisów (treśli, rysunków, symboli itp.) itd.

\LaTeX~jest zestawem instrukcji umożliwiających autorom skład i wydruk ich prac na najwyższym poziomie typograficznym. Do formatowania dokumentu \LaTeX~stosuje \TeX a (wymiawamy 'tech' -- greckie litery $\tau$, $\epsilon$, $\chi$). Korzystając z~systemu składu \LaTeX~mamy za zadanie przygotować jedynie tekst źródłowy, cały ciężar składania, formatowania dokumentu przejmuje na siebie system.

%---------------------------------------------------------------------------

\section{Cele pracy}
\label{sec:celePracy}


Celem poniższej pracy jest zapoznanie studentów z systemem \LaTeX~w zakresie umożliwiającym im samodzielne, profesjonalne złożenie pracy dyplomowej w systemie \LaTeX.

\subsection{Jakiś tytuł}

\subsubsection{Jakiś tytuł w subsubsection}


\subsection{Jakiś tytuł 2}

%---------------------------------------------------------------------------

\section{Zawartość pracy}
\label{sec:zawartoscPracy}

W rodziale~\ref{cha:pierwszyDokument} przedstawiono podstawowe informacje dotyczące struktury dokumentów w \LaTeX u. Alvis~\cite{Alvis2011} jest językiem 



















\chapter{Wstęp}
\label{cha:pierwszyDokument}




\begin{itemize}
\item Motywacja badań
\item Informacja o  'historii' badań QoE.
\item Co już w zadanej dziedzine zostało uzyskane.
\item Co poszerza moja  praca i jej cel.
\end{itemize}

Pierwsze obrazy wideo powstały już na początku XX wieku i opierały się na mechanicznie obracających się dyskach. Technologia ta istniała głownie w sferze badań akademickich i nie zdominowała rynku. Dopiero z wprowadzeniem cathode-ray tube (CRT), wraz z telewizją analogową, wideo zaczęło być wykorzystywane komercyjnie.  Z czasem rozwój technologii pozwolił na wprowadzenie telewizji cyfrowej, która zapewniała wyższą jakość obrazu oraz lepsze wykorzystanie zasobów. Wideo razem z audio okazały się również znakomitym środkiem wymiany informacji. Coraz częściej wykorzystywane do komunikacji w czasie rzeczywistym zastępując tradycyjne połączenie telefoniczne w biznesie oraz dla zwykłych użytkowników. Również rozwoj na rynku telefonów wspomógł powszechnoć wideo. W momencie kiedy praktycznie każdy aparat  zaczął posiadać kamerę, wideo zaczeło konkurować ze zdjęciami jako metoda na utrwalenia danej chwili. Codziennie tak rejestrowane obrazy są przekazywane do rodziny, znajomych oddalonych o tysiące kilometrów. Kolejnym przykładem kiedy wideo zastępuje tradycyjne formy przekazu są  blogi internetowe do tej pory prowadzone na zasadzie artykułów/postów, teraz zaczęły wykorzystywać wideo jako metodę  przekazu informacji.

Dzięki coraz większym przepustowością i szerokiemu dostępu do Internetu w najnowszych czasach wykreował się jeszcze inny trend sprawiający że obrazy wideo są bardziej popularne. Mowa tu o platformach streamingowych takich jak - YouTube, Netflix czy HBOgo. Pozwalają one użytkownikom na oglądanie od krótkich filmików, przez seriale, po pełnometrażowe filmy nawet w rozdzielczościach 4k. \par

%.Jako kolejny kierunek rozwoju wideo można potraktować powstałe w ostatnich latach platformy, typu Youtube, netflix czy HBOgo, pozwalające użytkownikom na streamowanie od krótkich %filmikow, przez seriale, po pełonometrazowe filmy nawet w rozdzielczochach 4k. \par

Wszystkie wymienione wyżej aspekty sprawiły, że wideo stało się codziennością w życiu większości ludzi.\par

Na obecnym etapie rozwoju technologii, oczekiwania odbiorcy co do jakości otrzymywanego wideo znacznie wzrosły. Na drugiej szali pozostają ograniczenia dotyczące medium i optymalnego wykorzystania zasobów po stronie klienta i serwera. Odnosząc się do powyższego istotną kwestią staje się monitorowanie jakości transmitowanego wideo i dostosowywanie go do potrzeb użytkownika. Jednak problem w ocenie jakości wideo jest tu o tyle trudny, że dotychczas najbardziej wiarygodnym wskaźnikiem jest tu opinia ludzka, nie powiązana(?żadnym agorytmem?) z technicznymi aspektami ?obrazu?\todo{, ciężka do powiązania technicznymi cechami wideo}. W niniejszej Pracy zostanie przedstawiony algorytm pozwalający na bardziej zautomatyzowaną ocenę jakości wideo w oparciu o metryki full-reference (FR) i no-reference (NR) oraz zaprezentowana zostanie wykorzystana metodologia badań.



\chapter{Wprowadzenie teorytyczne}
\label{cha:pierwszyDokument}


\section{Cechy statystyczne wideo}

\begin{itemize}
\item Ogólne informacje o wideo - czym jest, rodzaje.
\item Przedstawienie wybranych cech statycztycznych. ( wszystkich?) 
\end{itemize}


\section{Algorytmy uczenia maszynowego }
\label{cha:pierwszyDokument}

\begin{itemize}
\item Ogólne informacje uczeniu maszynowym/
\item Przedstawienie wybranych algorytmów 
\end{itemize}

\chapter{Metodologia badań}
\label{cha:pierwszyDokument}

\section{Dane}
\label{cha:pierwszyDokument}

\begin{itemize}
\item Wybrane narzędzia
\item Opis zebranych danych
\item Przedstawienie data flow(pobieranie-> czyszczenie->normalizacja->przygotowanie formatu dla modeli).
\item Wizualizacja danych
\end{itemize}


\section{Modele }
\label{cha:pierwszyDokument}

\begin{itemize}
\item Opis zastosowanych paramtrów/technik podczas trenowania.
\item Przedstawienie wyników 
\end{itemize}


\chapter{Analiza i wnioski }
\label{cha:pierwszyDokument}

\begin{itemize}
\item Interpretacja wyników
\item Opis innych czynników mogących zaburzyć ich prawdziwoć
\item Co nie zostało uwzględnione 
\end{itemize}


\chapter{Podsumowanie}
\label{cha:pierwszyDokument}

\begin{itemize}
\item Czy cel pracy został osiągniety.
\item Możliwoci rozbudowy
\end{itemize}
 





%\include{tests}



% itd.
% \appendix
% \include{dodatekA}
% \include{dodatekB}
% itd.

\printbibliography

\end{document}
