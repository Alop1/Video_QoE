\chapter{Wstęp}
\label{cha:pierwszyDokument}


Pierwsze obrazy wideo powstały już na początku XX wieku i opierały się na mechanicznie obracających się dyskach. Technologia ta istniała głownie w sferze badań akademickich i nie zdominowała rynku. Dopiero z wprowadzeniem cathode-ray tube (CRT), wraz z telewizją analogową, wideo zaczęło być wykorzystywane komercyjnie.  Z czasem rozwój technologii pozwolił na wprowadzenie telewizji cyfrowej, która zapewniała wyższą jakość obrazu oraz lepsze wykorzystanie zasobów. Wideo razem z audio okazały się również znakomitym środkiem wymiany informacji. Coraz częściej wykorzystywane do komunikacji w czasie rzeczywistym zastępując tradycyjne połączenie telefoniczne w biznesie oraz dla zwykłych użytkowników. Również rozwoj na rynku telefonów wspomógł powszechnoć wideo.\todo{dodać liczby} W momencie kiedy praktycznie każdy aparat  zaczął posiadać kamerę, wideo zaczeło konkurować ze zdjęciami jako metoda na utrwalenia danej chwili. Codziennie tak rejestrowane obrazy są przekazywane do rodziny, znajomych oddalonych o tysiące kilometrów. Kolejnym przykładem kiedy wideo zastępuje tradycyjne formy przekazu są  blogi internetowe do tej pory prowadzone na zasadzie artykułów/postów, teraz zaczęły wykorzystywać wideo jako metodę  przekazu informacji.

Dzięki coraz większym przepustowością i szerokiemu dostępu do Internetu w najnowszych czasach wykreował się jeszcze inny trend sprawiający że obrazy wideo są bardziej popularne. Mowa tu o platformach streamingowych takich jak - YouTube, Netflix czy HBOgo. Pozwalają one użytkownikom na oglądanie od krótkich filmików, przez seriale, po pełnometrażowe filmy nawet w rozdzielczościach 4k. \par

%.Jako kolejny kierunek rozwoju wideo można potraktować powstałe w ostatnich latach platformy, typu Youtube, netflix czy HBOgo, pozwalające użytkownikom na streamowanie od krótkich %filmikow, przez seriale, po pełonometrazowe filmy nawet w rozdzielczochach 4k. \par

Wszystkie wymienione wyżej aspekty sprawiły, że wideo stało się codziennością w życiu większości ludzi.\par

Na obecnym etapie rozwoju technologii, oczekiwania odbiorcy co do jakości otrzymywanego wideo znacznie wzrosły. Na drugiej szali pozostają ograniczenia dotyczące medium i optymalnego wykorzystania zasobów po stronie klienta i serwera. Odnosząc się do powyższego istotną kwestią staje się monitorowanie jakości transmitowanego wideo i dostosowywanie go do potrzeb użytkownika. Jednak problem w ocenie jakości wideo jest tu o tyle trudny, że dotychczas najbardziej wiarygodnym wskaźnikiem jest tu opinia ludzka, nie powiązana(?żadnym agorytmem?) z technicznymi aspektami ?obrazu?\todo{, ciężka do powiązania technicznymi cechami wideo}. W niniejszej Pracy zostanie przedstawiony algorytm pozwalający na bardziej zautomatyzowaną ocenę jakości wideo w oparciu o metryki full-reference (FR) i no-reference (NR) oraz zaprezentowana zostanie wykorzystana metodologia badań.



\chapter{Wprowadzenie teorytyczne}
\label{cha:pierwszyDokument}

W niniejszym rodziale przedsatawiono najważniejsze z zagadnienia dotyczące przeprowadzonych badań. Opisane one zostały w sposób pozwalający czytelnikowi na odpowiednie zrozumienie dalszej częci pacy, pomijając niezwiązane szczegóły. Pierwsza częć rozdziału dotyczy tematu wideo. Przedsawiono jego definicję, oraz wybrane cechy statystyczne  biorące udział w trakcie badań .W kolejnej częsci przedstawiono zagadnienia z obszaru uczenia maszynowego. Wyjaniona została jej ogólna koncepcja, a następnie opisano użyte algorytmy.


\section{Cechy statystyczne wideo}

<definicja ciężko znaleć :( >
Wideo jest formą elektronicznego zapisu sygnału wizji (analogowego bądź cyfrowego). W swojej surowej postaci jest to sekwencja pojedyńczych ramek. Takie pliki  zajmują bardzo dużo przestrzeni dyskowej. Przykładowo obraz o rodzielcozci 1920x1080 z 24 ramkami na sekunde o długoći 30 sekund zajmuje ponad \todo{sprawdzic ile dokladnie}10 GB w pamięci komutera. Tak duże rozmiary znacząco organiczają możliwoci przechowywania i transmisji dla zwyklych użykowników. Dlatego praktycznie każdy plik wideo wykorzystuje kodeki, czyli pewne ustandaryzowane zasady kompresji/dekompresji. Do najpopularniejszych należą H.265 i  H.264 powszechnie używany w Interneci do transmitowania multimediów \cite{video_codecs}. Dodatkowo wideo można charakteryzować poprzez wiele innych wskaźników\todo{to moze byc nowy akapit}. W poniżej zostały opisane te chechy statystyczne,  które brały udział w  badaniu.

\begin{itemize}
\item Rozdzielczoć -  miara okrelającą rozmiar ramki. Jenostką są pixele.  Podawana jest zazwyczaj w nastepujący sposób: szerokosc x wyskosc. Do badań zostały użyto wideo po rozdzielczosciach: 3840x2160, 1920x1080, 704x576, 640x480, 352x288.
\item Klatki na sekunde( ang.frames per second, fps ) - liczba ramek wywietlonych w czasie sekundy. W telewizji jest to 25 ramek na sekunde. Do badań użyto: 60, 30,  25, 24 fps  
\item blockiness
\item spatialact
\item letterbox 
\item pillarbox 
\item blockloss 
\item blur 
\item temporalac
\item blockout 
\item freezing 
\item exposure 
\item contrast 
\item brightness
\item interlace 
\item noise 
\item slice 
\item flickering
\end{itemize}

opis metryk i ich wymienienie  w oparciu o https://ieeexplore.ieee.org/document/5506331



\begin{itemize}
\item Ogólne informacje o wideo - czym jest, rodzaje.
\item Przedstawienie wybranych cech statycztycznych. ( wszystkich?) 
\end{itemize}


\section{Algorytmy uczenia maszynowego }
\label{cha:pierwszyDokument}

\begin{itemize}
\item Ogólne informacje uczeniu maszynowym/
\item Przedstawienie wybranych algorytmów 
\end{itemize}

\chapter{Metodologia badań}
\label{cha:pierwszyDokument}

\section{Dane}
\label{cha:pierwszyDokument}

\begin{itemize}
\item Wybrane narzędzia
\item Opis zebranych danych
\item Przedstawienie data flow(pobieranie-> czyszczenie->normalizacja->przygotowanie formatu dla modeli).
\item Wizualizacja danych
\end{itemize}


\section{Modele }
\label{cha:pierwszyDokument}

\begin{itemize}
\item Opis zastosowanych paramtrów/technik podczas trenowania.
\item Przedstawienie wyników 
\end{itemize}


\chapter{Analiza i wnioski }
\label{cha:pierwszyDokument}

\begin{itemize}
\item Interpretacja wyników
\item Opis innych czynników mogących zaburzyć ich prawdziwoć
\item Co nie zostało uwzględnione 
\end{itemize}


\chapter{Podsumowanie}
\label{cha:pierwszyDokument}

\begin{itemize}
\item Czy cel pracy został osiągniety.
\item Możliwoci rozbudowy
\end{itemize}
 




