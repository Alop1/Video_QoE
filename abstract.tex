\begin{abstract}
	Dzisiejsze czasy coraz mocniej nastawione \emph{user experience} oraz wysoki stopień optymalizacji zasobów. Mimo wielu starań badaczy metryki oceny jakości wideo w kontekście wyżej wymienionych warunków, nie spełnia oczekiwań. Przyczyną jest ludzka percepcja. To co człowiek intuicyjnie stwierdzi jako w oczywisty sposób lepsze, wciąż w wielu przypadkach jest niemożliwe do osądzenia dla maszyny. Niniejsza praca przedstawiła algorytm, pozwalający dokonać tej oceny, tak aby bardziej zbliżyć się do oceny ludzkiej. Do badań zebrano 5 baz plików wideo, z których zostały wyekstrahowane cechy. Następnie dane odpowiednio przygotowano do tworzenia modeli. Modele natomiast powstały w oparciu o cztery algorytmy uczenia maszynowego. Na podstawie oceny trafności predykcji, z tych modeli, dokonano wyboru optymalnego rozwiązania, które w ostatecznym podsumowaniu stanowi algorytm oceny jakości wideo.	
\end{abstract}
